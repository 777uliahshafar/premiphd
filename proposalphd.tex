%%%%%%%%%%%%%%%%%%%%%%%%%%%%%%%%%%%%%%%%%
% Simple Article
% Integrated article template with simple for make4ht
% LaTeX Class
% Version 1.0 (10/11/20)
%
% This class originates by:
% Vel and  Nicolas Diaz
%
% Authors:
% Muhammad Uliah Shafar
%
%
% Free License:
%
%
%%%%%%%%%%%%%%%%%%%%%%%%%%%%%%%%%%%%%%%%%
\documentclass[11pt]{simart} % Font size (can be 10pt, 11pt or 12pt)

%----------------------------------------------------------------------------------------
%	TITLE SECTION
%----------------------------------------------------------------------------------------
% MAIN TITLE SECTION
\title{
\textbf{Tipikal Ruang Publik yang Terbaru di Kota Parepare.} \\
\textbf{{Lorem Ipsum Lorem Ipsum \\}}
} % Title and subtitle
%\date{\textbf{\DTMtoday}}
\date{\textbf{\today}}
\author{
\begin{tabular}{@{}ll@{}}
	Nama & : Muhammad Uliah Shafar\\
	NIM & : 21020119420029\\
\end{tabular}
}

%----------------------------------------------------------------------------------------
% OTHER TITLE SECTION

%\title{\textbf{Sistem Sarana dan Prasarana Jl. Pinggir Laut} \\ {\Large\itshape Infrastructure of Waterfront Parepare City}} % Title and subtitle

%\author{\textbf{Uliah Shafar} \\ \textit{Universitas Diponegoro}} % Author and institution

%\date{\today} % Date, use \date{} for no date

%----------------------------------------------------------------------------------------



\begin{document}
\thispagestyle{empty}
	\begin{center}
		\begin{huge}
			\bf{Lorem Ipsum Lorem Ipsum}
		\end{huge}

		\vspace{20pt}
		\includegraphics[width=0.35\textwidth]{logo} \\

		\vspace*{35pt}

		\begin{large}
		\textbf{Lorem Ipsum Lorem Ipsum} \\
			Lorem Ipsum Lorem\\

		\vspace{20pt}
		\textbf{oleh\\
			\vspace{20pt}
			Muhammad Uliah Shafar\\21020119420029}\\

			\vspace{20pt}
		Dosen: \\
		\textbf{Lorem Ipsum}\\



		\vspace{60pt}
		\textbf{PROGRAM STUDI MAGISTER ARSITEKTUR DEPARTEMEN ARSITEKTUR\\
			UNIVERSITAS DIPONEGORO\\
			SEMARANG\\
			2020
		}
		\end{large}
	\end{center}
\clearpage

\maketitle % Print the title section

%----------------------------------------------------------------------------------------
%	ABSTRACT AND KEYWORDS
%----------------------------------------------------------------------------------------
\begin{abstract}
Lorem ipsum dolor sit amet, consectetur adipiscing elit. Curabitur eget faucibus dolor. In posuere, est nec mollis ultrices, ante arcu tristique odio, et rhoncus tortor enim vitae lectus. Aenean auctor enim tempor risus vulputate finibus. Ut quis molestie ex, ut fringilla mauris. Suspendisse ornare sapien nec neque placerat dignissim. Sed vehicula feugiat dolor et blandit. Maecenas convallis diam a lacus faucibus faucibus. Quisque efficitur velit quis lorem consectetur, ac dictum est egestas.


\end{abstract}

\hspace*{3.6mm}\textit{Keywords:} Lorem, Ipsum % Keywords

\vspace{30pt} % Vertical whitespace between the abstract and first section

%----------------------------------------------------------------------------------------
%	ESSAY BODY
%----------------------------------------------------------------------------------------
\section{Research Objective and Questions}
\subsection{Research Objective}

%The role of developing waterfront in improving identify of parepare town as recreational town

\begin{enumerate}
\item The proposed research seeks to analyze the importance of developing three waterfronts landscape in arrangement of the place identity. Therefore, this research project will investigate what are characteristic of waterfront landscape in Parepare town identified in the study area.
\item This research proposal will investigate the character of developing waterfront landscape of recreational Parepare city from place, contemporary, and recreational significance point of views.
\item The proposed study will focus on creating recomendation to improvement of waterfronts development in enhancing the identity of Recrational City.

\end{enumerate}

\subsection{Research Question}

\subsubsection{Draft}
What are the characterstics of Parepare waterfronts related to recreational city context?

Waterfront is one of the interesting place in the city these days. Although it had experienced several function changes, like industrial, cultural, and business function, many cities centralize its development to public open space in the waterfront, including Parepare. Waterfront in Parepare has improved significantly in the past decade. Its development was one of the result of city major's effort in order to make Parepare become more recreational. This improvement is also caused by the increasing of population density. Many new settlements have been growing in the suburb area. This phenomenon has raised the need of better public open space in the city center.

Cities become interest with its functional to bring public space to attract people from city centre.

Until it become the public space,

In the past decade, Parepare is focus on infrastructure development.

Tepi laut menjadi sumber kehidupan pada sebuah daerah yang memiliki garis pantai yang panjang.

Selayaknya air, tepi laut menjadi sumber kehidupan pada sebuah daerah yang memiliki garis pantai yang signifikan. Tidak seperti sebuah daerah yang tidak memiliki garis pantai, terpaksa menggantungkan kehidupan pada sumber lainnya seperti.

Tepi laut adalah sebuah tempat yang berada di pinggiran laut.

menyebut terbentang 30 meter dari garis pantai.

\section{Typical New Public Space in Parepare City}
Public space has become a intriguing discussion on urban planning in the last 100 years. One of the reason is the increasing population in the city that hope the open space. This increasing population is marked by the massive development of residence, the density of high way, the rising of business and economy sphere that has occured in Parepare. The increasing population density put stress on individu to go through their live in every single day. People are immedietyly seeking a place where they can heal from the hustle and bustle of daily routine and reduce the stress as a result of urban living.





\section{Tipikal Ruang Publik yang Terbaru di Kota Parepare}
%munculnya ruang bpublik
Ruang publik telah menjadi momok pembicaraan dalam perencanaan kota dalam 100 tahun terakhir ini. Satu alasannya adalah peningkatan populasi yang kian membesar pada sebuah kota yang mengharapkan ruang yang terbuka. Peningkatan populasi tersebut ditandai dengan adanya pembangunan perumahan yang besar-besaran, keramaian di jalan raya, meningkatknya sektor bisnis dan perekonomian  seperti yang terjadi di Parepare. Peningkatan populasi ini memberi tekanan pada setiap individu dalam menjalani kehidupan sehari-hari.
Orang-orang kemudian berbondong-bondong mencari tempat untuk penyembuhan \textit{(healing)} untuk keluar dari hiruk pikuk rutinitas sehari-hari dan menghilangkan tekanan akibat dari suasana perkotaan.

%ruang publik pd perkotaan
Ruang publik adalah sebuah tempat yang mewadahi segala aktivitas perorangan atau perkelompok dalam mencari kesenangan dan hal positif lainnya \citep{sadat2012}.
Diluar dari ruang publik, banyak orang memilih untuk sepenuhnya berekreasi ditempat lain yang menawarkan rekreasi khusus seperti pantai, waterbom, \textit{themepark}, pegunungan, dan kampung. Namun tidak sedikit yang memilih \textit{healing} secara sederhana dalam sebuah ruang publik perkotaan yang menawarkan kehidupan perkotaan yang dibungkus oleh aktivitas yang menyenangkan.
Ruang terbuka publik (rtp) tercipta untuk mereka yang mencari pusat rekreasi  dengan akses yang gratis, akses yang mudah dan tidak memerlukan banyak waktu tempuh. Sehingga ruang publik menjadi primadona bagi mereka yang tidak menyukai kerumitan untuk berwisata. Meskipun ruang publik terlihat sederhana, banyak potensi-pontesi ruang publik yang bernilai tinggi apabila orang menjelajahinya lebih dalam.

%potensi ruang publik
Ruang publik sepatutnya tidak terikat oleh batas ruang dan waktu. Itu memberi ruang untuk setiap golongan dalam jumlah berapapun untuk memaksimalkan ruang publik sebagai tempat untuk berekreasi. Walaupun sifatnya satu ruang untuk semua orang, dimana kita berbagi dengan orang asing secara sukarela \citep{sadat2012}. Ruang publik berkesempatan untuk mengakomodasi segala bentuk aktivitas didalamnya. Sehingga tujuan dominan dari ruang publik yaitu mensejahterakan masyarakat dapat teruwujud.

Banyak yang bergumam tentang ruang publik yang seutuhnya.




%rtp secara umum


% Peningkatan di parepare
Kota ini dalam beberapa tahun terakhir menunjukkan peningkatan dalam segi pelayanan bidang dan jasa. Sehingga mendorong terjadinya urbanisasi yang cukup tinggi.

Kualitas hidup seseorang ditentukan oleh elemen ruang publik dari sebuah kota.


Meningkatnya pembangunan perumahan di kota Parepare, mendorong ketersediaan ruang publik yang memenuhi kebutuhan masyarakat Parepare.



%Therefore, project of research will investigate what are characters and elements of developing waterfronts in the context of Recreational Parepare town.
%\item This research proposal examine the impact of developing waterfronts landscape towards the growing recreational city of Parepare. Then, classify what are growing waterfronts landscape charateristic that can be identified in the area of study.
%\item The proposed research will focus on the element of growing waterfronts landscape that can be maintainted to increase the value of recreational Parepare town.

\subsection{Research Questions}

The proposed research seeks to analyze developing three waterfronts landscape in arrangement of the place identity. Therefore, project of research will investigate what are characters and elements of developing waterfronts in the context of Recreational Parepare town.
This research proposal examines the impact of developing waterfronts landscape towards the growing recreational city of Parepare. Then, classify what are growing waterfronts landscape charateristic that can be identified in the area of study.
The proposed research will focus on the element of growing waterfronts landscape that can be maintainted to increase the value of recreational Parepare town.

% draft words
This research proposal is seeking to understand
The proposed research seeks to


Waterfront in Parepare has undergone rapid development.

How close bj habibie image on waterfront development parepare.

Transformation of waterfront parepare toward the city development.

Are the waterfront valued the same by people.

Waterfront character as identity representative of the Parepare city
* does waterfront should be designed following city branding.

what the connection of waterfront in parepare in terms of image of the city.

The role of devoloping waterfront in improving identify of parepare city as recreational town



this file powered by \citep{einstein} and \citep{latexcompanion}. It can also be found in \cite{knuthwebsite}.

\begin{figure}[htpb]
	\centering
	\includegraphics[width=0.8\textwidth]{placeholder}
	\caption{placeholder.jpg}
	\caption*{Sumber: Dokumen Pribadi, 2020}
	\label{fig:placeholder-jpg}
\end{figure}
\lipsum[1-3]

\subsection{Latar Belakang}

\lipsum[4-5]


\subsubsection{Definisi}

\lipsum[6-7]
\begin{table}[h] % [h] forces the table to be output where it is defined in the code (it suppresses floating)
	\caption{Example table.}
	\centering
	\begin{tabular}{l l r}
		\toprule
		\multicolumn{2}{c}{Name} \\
		\cmidrule(r){1-2}
		First Name & Last Name & Grade \\
		\midrule
		John & Doe & $7.5$ \\
		Richard & Miles & $5$ \\
		\bottomrule
	\end{tabular}
\end{table}



\subfile{subfiles/subfile.tex}

\section{Metodologi Penelitian}

\lipsum[15-19]

%\begin{figure}[htbp]
%\centering
%\begin{subfigure}{6cm}
%\centering\includegraphics[width=5cm]{placeholder.jpg}
%\caption{}
%\end{subfigure}%
%\begin{subfigure}{6cm}
%\centering\includegraphics[width=5cm]{placeholder.jpg}
%\caption{}
%\end{subfigure}\vspace{10pt}
%
%\caption{Lorem Ipsum}
%\label{}
%\end{figure}

%----------------------------------------------------------------------------------------
%	BIBLIOGRAPHY
%----------------------------------------------------------------------------------------

\bibliographystyle{apalike}

\bibliography{biblio.bib}



\end{document}
