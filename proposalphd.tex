%%%%%%%%%%%%%%%%%%%%%%%%%%%%%%%%%%%%%%%%%
% Simple Article
% Integrated article template with simple for make4ht
% LaTeX Class
% Version 1.0 (10/11/20)
%
% This class originates by:
% Vel and  Nicolas Diaz
%
% Authors:
% Muhammad Uliah Shafar
%
%
% Free License:
%
%
%%%%%%%%%%%%%%%%%%%%%%%%%%%%%%%%%%%%%%%%%
\documentclass[11pt]{simart} % Font size (can be 10pt, 11pt or 12pt)

%----------------------------------------------------------------------------------------
%	TITLE SECTION
%----------------------------------------------------------------------------------------
% MAIN TITLE SECTION
\title{
\textbf{Tipikal Ruang Publik yang Terbaru di Kota Parepare.} \\
\textbf{{Lorem Ipsum Lorem Ipsum \\}}
} % Title and subtitle
%\date{\textbf{\DTMtoday}}
\date{\textbf{\today}}
\author{
\begin{tabular}{@{}ll@{}}
	Nama & : Muhammad Uliah Shafar\\
	NIM & : 21020119420029\\
\end{tabular}
}

%----------------------------------------------------------------------------------------
% OTHER TITLE SECTION

%\title{\textbf{Sistem Sarana dan Prasarana Jl. Pinggir Laut} \\ {\Large\itshape Infrastructure of Waterfront Parepare City}} % Title and subtitle

%\author{\textbf{Uliah Shafar} \\ \textit{Universitas Diponegoro}} % Author and institution

%\date{\today} % Date, use \date{} for no date

%----------------------------------------------------------------------------------------



\begin{document}
\thispagestyle{empty}
	\begin{center}
		\begin{huge}
			\bf{Lorem Ipsum Lorem Ipsum}
		\end{huge}

		\vspace{20pt}
		\includegraphics[width=0.35\textwidth]{logo} \\

		\vspace*{35pt}

		\begin{large}
		\textbf{Lorem Ipsum Lorem Ipsum} \\
			Lorem Ipsum Lorem\\

		\vspace{20pt}
		\textbf{oleh\\
			\vspace{20pt}
			Muhammad Uliah Shafar\\21020119420029}\\

			\vspace{20pt}
		Dosen: \\
		\textbf{Lorem Ipsum}\\



		\vspace{60pt}
		\textbf{PROGRAM STUDI MAGISTER ARSITEKTUR DEPARTEMEN ARSITEKTUR\\
			UNIVERSITAS DIPONEGORO\\
			SEMARANG\\
			2020
		}
		\end{large}
	\end{center}
\clearpage

\maketitle % Print the title section

%----------------------------------------------------------------------------------------
%	ABSTRACT AND KEYWORDS
%----------------------------------------------------------------------------------------
\begin{abstract}
Lorem ipsum dolor sit amet, consectetur adipiscing elit. Curabitur eget faucibus dolor. In posuere, est nec mollis ultrices, ante arcu tristique odio, et rhoncus tortor enim vitae lectus. Aenean auctor enim tempor risus vulputate finibus. Ut quis molestie ex, ut fringilla mauris. Suspendisse ornare sapien nec neque placerat dignissim. Sed vehicula feugiat dolor et blandit. Maecenas convallis diam a lacus faucibus faucibus. Quisque efficitur velit quis lorem consectetur, ac dictum est egestas.


\end{abstract}

\hspace*{3.6mm}\textit{Keywords:} Lorem, Ipsum % Keywords

\vspace{30pt} % Vertical whitespace between the abstract and first section

%----------------------------------------------------------------------------------------
%	ESSAY BODY
%----------------------------------------------------------------------------------------
\section{Research Objective and Questions}
Typical New Public Space in Parepare City

\subsection{Research Objectives}

\subsection{Research Questions}



\section{Introduction of Research Proposal}

\subsection{Background of research}

%the emerge of public space
Public space has become a provocative discussion on urban planning in the last 100 years. It was led by an increasing of population in the city which hope for the open space.
The increasing population is marked by the massive development of residence, the density of highway, the rising of business and economy sphere that has occurred in Parepare. It put stress on individual to go through their lives in every single day. People start seeking a place where they can heal from the hustle and bustle of daily routine and reduce the stress as a result of urban living.

%public space in the city
Public space is a place that provide any kind of individual or group  activities in pursuit of positive things \citep{sadat2012}.\cite{sadat2012} explain a large square, waterfront, street, and traditional bazaar as a form of a public space.
People might find public space as a place for recreation, working, business, social, cultural and heritage. \cite{sadat2012,akkar2007} have collected the function of public space broadly consist of physical, ecological, physiological, social, political, economic, symbolic and aesthetics roles.
Moreover, one public space can extend to other purpose. For example, Maliboro corridor in Jogjakarta have business and recreation function. Beside that, it also depict some cultural and heritage element that until now are preserved.

People begin to notice the importance of public space in the city when they cannot find the relaxing, calm, and comfortable place nearby and easily. Sometimes, the importance become more striking when the place was used in the daily basis. It can be a street, or corridor, bazaar, plaza surrounded by building where people especially workers, pedestrian, and picnicker are doing their own thing, such as shopping exactly at front store, talking while walking, or just passing through for work.
The activities took place around strengthen the importance of public space. But they also show contribution determining the quality of urban lives and mirror the state of neighborhood.

% people views
Public space then have their own characteristic that people are most enjoyed. Researcher found that public space attract more people because it has adequate facilities which they can use it for relaxation, sport and interaction with their friends.
\cite{uliah} argue that people prefer public space that have outstanding setting rather than just natural setting or without many facilities. Regardless the preference of most people, multiple variation of public space are emerging in few place of the city.
This may because of the diversity dwellers around those public space and their unique needs towards a space \citep{ahmadi2009}.

\cite{ahmadi2009} say that the public space has a different value. Although it has unique value, sometimes adjacent public spaces express the same common purpose. One of significant purpose of most public space is economic \citep{akkar2007}, which has evolved since long time. But, some of them is following the present-day demands like the physiological and physical which are results of hectic urban lives. So those different purpose, one of them will adhere in the identity of the city.

Parepare has owned few of public space, one of the greatest is Andi Makassau Park which has undergone some revitalization. Its recent revitalization happen in an early 2022. It brings some facilities for sport, health and other improved facilities such as bench, toilet, stage, maintained grass.
This improvement has altered the city identity. Furthermore, specifically, it increase the physiological well-being through the availability of facility. It shows that more or less this public space has pictured some states of society who are active in sports and fitness. In addition to that, This park also successfully have arranged several events include Independence day ceremony, gymnastic, and other events which were attracted vendors to do economic activity. Therefore, the place has shown other than just physical or physiological role of urban space, yet it also shows an economy part of Andi Makassau Park.

The role Andi makassau play as public space might spread across other public space. For instance, the near public psace like Parepare Beach, Tonrangeng Waterfront, and Syariah Small Park. These public space have a number of roles similar to  Andi Makassau Park. The role they have played in the city will have contribution to the urban people. For example, instead, Parepare beach provide a jogging track, it gives a place for visitor to swimming, eating and even just having interaction. Then places eventually bring some benefit to people.

Among all the roles that can be played public space, it is important to understand a general image that public space give to the city. One of the reason is that public space might represent the element which is containing a value of higher order \citep{akkar2007}.  For example, sacred or symbolic meeting. Habibie sculpture comes up as an example of symbolic meeting. This sculpture is placed in the corner of Andi Makassau Park. It has been popular among the tourism and visitor of Parepare city and built around a decade.
The name of statue was inspired by the name of the third Indonesia president. The sculpture is purposely built to attract visitor from all around South Sulawesi province and outside and also strengthen the image of Parepare as tourism city.


A city is supposed to be a brand of







The use of jogging track facility in the Andi Makassau Park, for example, has indicated that there are numerous demands for exercise in the city \citep{shafar2021}.


The diversity of public space in the city indeed rise a question that is what do the public space in the city as whole picture for the city.

This question may benefit to understand what people is like in Parepare.





terdapat permintaan yang banyak terhadap olahraga
there are demands for exercise.

This research try to understand what is the common style of public space.











%--------------------------------------------------------------------------------------
% Draft
%
%--------------------------------------------------------------------------------------
\begin{comment}
%The role of developing waterfront in improving identify of parepare town as recreational town

\begin{enumerate}
\item The proposed research seeks to analyze the importance of developing three waterfronts landscape in arrangement of the place identity. Therefore, this research project will investigate what are characteristic of waterfront landscape in Parepare town identified in the study area.
\item This research proposal will investigate the character of developing waterfront landscape of recreational Parepare city from place, contemporary, and recreational significance point of views.
\item The proposed study will focus on creating recomendation to improvement of waterfronts development in enhancing the identity of Recrational City.

\end{enumerate}

\subsection{Research Question}

\subsubsection{Draft}
What are the characterstics of Parepare waterfronts related to recreational city context?

Waterfront is one of the interesting place in the city these days. Although it had experienced several function changes, like industrial, cultural, and business function, many cities centralize its development to public open space in the waterfront, including Parepare. Waterfront in Parepare has improved significantly in the past decade. Its development was one of the result of city major's effort in order to make Parepare become more recreational. This improvement is also caused by the increasing of population density. Many new settlements have been growing in the suburb area. This phenomenon has raised the need of better public open space in the city center.

Cities become interest with its functional to bring public space to attract people from city centre.

Until it become the public space,

In the past decade, Parepare is focus on infrastructure development.

Tepi laut menjadi sumber kehidupan pada sebuah daerah yang memiliki garis pantai yang panjang.

Selayaknya air, tepi laut menjadi sumber kehidupan pada sebuah daerah yang memiliki garis pantai yang signifikan. Tidak seperti sebuah daerah yang tidak memiliki garis pantai, terpaksa menggantungkan kehidupan pada sumber lainnya seperti.

Tepi laut adalah sebuah tempat yang berada di pinggiran laut.

menyebut terbentang 30 meter dari garis pantai.



\section{Tipikal Ruang Publik yang Terbaru di Kota Parepare}
%munculnya ruang bpublik
Ruang publik telah menjadi momok pembicaraan dalam perencanaan kota dalam 100 tahun terakhir ini. Satu alasannya adalah peningkatan populasi yang kian membesar pada sebuah kota yang mengharapkan ruang yang terbuka. Peningkatan populasi tersebut ditandai dengan adanya pembangunan perumahan yang besar-besaran, keramaian di jalan raya, meningkatknya sektor bisnis dan perekonomian  seperti yang terjadi di Parepare. Peningkatan populasi ini memberi tekanan pada setiap individu dalam menjalani kehidupan sehari-hari.
Orang-orang kemudian berbondong-bondong mencari tempat untuk penyembuhan \textit{(healing)} untuk keluar dari hiruk pikuk rutinitas sehari-hari dan menghilangkan tekanan akibat dari suasana perkotaan.

%ruang publik pd perkotaan
Ruang publik adalah sebuah tempat yang mewadahi segala aktivitas perorangan atau perkelompok dalam mencari kesenangan dan hal positif lainnya \citep{sadat2012}.
Diluar dari ruang publik, banyak orang memilih untuk sepenuhnya berekreasi ditempat lain yang menawarkan rekreasi khusus seperti pantai, waterbom, \textit{themepark}, pegunungan, dan kampung. Namun tidak sedikit yang memilih \textit{healing} secara sederhana dalam sebuah ruang publik perkotaan yang menawarkan kehidupan perkotaan yang dibungkus oleh aktivitas yang menyenangkan.
Ruang terbuka publik (rtp) tercipta untuk mereka yang mencari pusat rekreasi  dengan akses yang gratis, akses yang mudah dan tidak memerlukan banyak waktu tempuh. Sehingga ruang publik menjadi primadona bagi mereka yang tidak menyukai kerumitan untuk berwisata. Meskipun ruang publik terlihat sederhana, banyak potensi-pontesi ruang publik yang bernilai tinggi apabila orang menjelajahinya lebih dalam.

%potensi ruang publik
Ruang publik sepatutnya tidak terikat oleh batas ruang dan waktu. Itu memberi ruang untuk setiap golongan dalam jumlah berapapun untuk memaksimalkan ruang publik sebagai tempat untuk berekreasi. Walaupun sifatnya satu ruang untuk semua orang, dimana kita berbagi dengan orang asing secara sukarela \citep{sadat2012}. Ruang publik berkesempatan untuk mengakomodasi segala bentuk aktivitas didalamnya. Sehingga tujuan dominan dari ruang publik yaitu mensejahterakan masyarakat dapat teruwujud.

Banyak yang bergumam tentang ruang publik yang seutuhnya.

%rtp secara umum

% Peningkatan di parepare
Kota ini dalam beberapa tahun terakhir menunjukkan peningkatan dalam segi pelayanan bidang dan jasa. Sehingga mendorong terjadinya urbanisasi yang cukup tinggi.

Kualitas hidup seseorang ditentukan oleh elemen ruang publik dari sebuah kota.

Meningkatnya pembangunan perumahan di kota Parepare, mendorong ketersediaan ruang publik yang memenuhi kebutuhan masyarakat Parepare.


%--------------------
%Therefore, project of research will investigate what are characters and elements of developing waterfronts in the context of Recreational Parepare town.
%\item This research proposal examine the impact of developing waterfronts landscape towards the growing recreational city of Parepare. Then, classify what are growing waterfronts landscape charateristic that can be identified in the area of study.
%\item The proposed research will focus on the element of growing waterfronts landscape that can be maintainted to increase the value of recreational Parepare town.

\subsection{Research Questions}

The proposed research seeks to analyze developing three waterfronts landscape in arrangement of the place identity. Therefore, project of research will investigate what are characters and elements of developing waterfronts in the context of Recreational Parepare town.
This research proposal examines the impact of developing waterfronts landscape towards the growing recreational city of Parepare. Then, classify what are growing waterfronts landscape charateristic that can be identified in the area of study.
The proposed research will focus on the element of growing waterfronts landscape that can be maintainted to increase the value of recreational Parepare town.

% draft words
This research proposal is seeking to understand
The proposed research seeks to


Waterfront in Parepare has undergone rapid development.

How close bj habibie image on waterfront development parepare.

Transformation of waterfront parepare toward the city development.

Are the waterfront valued the same by people.

Waterfront character as identity representative of the Parepare city
* does waterfront should be designed following city branding.

what the connection of waterfront in parepare in terms of image of the city.

The role of devoloping waterfront in improving identify of parepare city as recreational town



this file powered by \citep{einstein} and \citep{latexcompanion}. It can also be found in \cite{knuthwebsite}.

\begin{figure}[htpb]
	\centering
	\includegraphics[width=0.8\textwidth]{placeholder}
	\caption{placeholder.jpg}
	\caption*{Sumber: Dokumen Pribadi, 2020}
	\label{fig:placeholder-jpg}
\end{figure}
\lipsum[1-3]

\subsection{Latar Belakang}

\lipsum[4-5]


\subsubsection{Definisi}

\lipsum[6-7]
\begin{table}[h] % [h] forces the table to be output where it is defined in the code (it suppresses floating)
	\caption{Example table.}
	\centering
	\begin{tabular}{l l r}
		\toprule
		\multicolumn{2}{c}{Name} \\
		\cmidrule(r){1-2}
		First Name & Last Name & Grade \\
		\midrule
		John & Doe & $7.5$ \\
		Richard & Miles & $5$ \\
		\bottomrule
	\end{tabular}
\end{table}



\subfile{subfiles/subfile.tex}

\section{Metodologi Penelitian}

\lipsum[15-19]

%\begin{figure}[htbp]
%\centering
%\begin{subfigure}{6cm}
%\centering\includegraphics[width=5cm]{placeholder.jpg}
%\caption{}
%\end{subfigure}%
%\begin{subfigure}{6cm}
%\centering\includegraphics[width=5cm]{placeholder.jpg}
%\caption{}
%\end{subfigure}\vspace{10pt}
%
%\caption{Lorem Ipsum}
%\label{}
%\end{figure}

\end{comment}
%----------------------------------------------------------------------------------------
%	BIBLIOGRAPHY
%----------------------------------------------------------------------------------------

\bibliographystyle{apalike}

\bibliography{biblio.bib}



\end{document}
