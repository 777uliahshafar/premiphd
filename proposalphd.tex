%%%%%%%%%%%%%%%%%%%%%%%%%%%%%%%%%%%%%%%%%
% Simple Article
% Integrated article template with simple for make4ht
% LaTeX Class
% Version 1.0 (10/11/20)
%
% This class originates by:
% Vel and  Nicolas Diaz
%
% Authors:
% Muhammad Uliah Shafar
%
%
% Free License:
%
%
%%%%%%%%%%%%%%%%%%%%%%%%%%%%%%%%%%%%%%%%%
\documentclass[12pt]{simart} % Font size (can be 10pt, 11pt or 12pt)

%----------------------------------------------------------------------------------------
%	TITLE SECTION
%----------------------------------------------------------------------------------------
% MAIN TITLE SECTION
\title{
\textbf{Tipikal Ruang Publik yang Terbaru di Kota Parepare.} \\
\textbf{{Lorem Ipsum Lorem Ipsum \\}}
} % Title and subtitle
%\date{\textbf{\DTMtoday}}
\date{\textbf{\today}}
\author{
\begin{tabular}{@{}ll@{}}
	Nama & : Muhammad Uliah Shafar\\
	NIM & : 21020119420029\\
\end{tabular}
}

%----------------------------------------------------------------------------------------
% OTHER TITLE SECTION

%\title{\textbf{Sistem Sarana dan Prasarana Jl. Pinggir Laut} \\ {\Large\itshape Infrastructure of Waterfront Parepare City}} % Title and subtitle

%\author{\textbf{Uliah Shafar} \\ \textit{Universitas Diponegoro}} % Author and institution

%\date{\today} % Date, use \date{} for no date

%----------------------------------------------------------------------------------------



\begin{document}

\maketitle % Print the title section

%----------------------------------------------------------------------------------------
%	ABSTRACT AND KEYWORDS
%----------------------------------------------------------------------------------------
\begin{abstract}
Lorem ipsum dolor sit amet, consectetur adipiscing elit.

\end{abstract}

\hspace*{3.6mm}\textit{Keywords:} Lorem, Ipsum % Keywords

\vspace{30pt} % Vertical whitespace between the abstract and first section

%----------------------------------------------------------------------------------------
%	ESSAY BODY
%----------------------------------------------------------------------------------------

\section{Latar belakang}

Ruang publik menjadi pembicaraan yang hangat dalam perencanaan kota dalam 100 tahun terakhir. Populasi yang terus meningkat menyebabkan adanya permintaan terhadap ruang terbuka.
Peningkatan ini ditandai dengan adanya pembangunan perumahan yang besar-besaran, kepadatan kendaraan di jalan raya, serta meningkatknya sektor bisnis dan perekonomian. Peningkatan populasi ini memberi tekanan pada setiap individu dalam menjalani kehidupan sehari-hari.
Orang-orang kemudian berbondong-bondong mencari tempat untuk penyembuhan \textit{(restorative)} untuk keluar dari hiruk pikuk rutinitas sehari-hari dan menghilangkan tekanan akibat dari suasana perkotaan.

%definisi ruang publik
Ruang publik adalah sebuah tempat yang mewadahi segala aktivitas perorangan atau perkelompok dalam mencari kesenangan dan hal positif lainnya \citep{sadat2012}.
\cite{sadat2012} menjelaskan bahwa taman, alun-alun, tepi laut, dan jalan adalah bentuk dari sebuah ruang publik. Berdasarkan fungsinya, ruang publik dapat menjadi tempat untuk rekreasi, bekerja, bisnis, sosial budaya dan peninggalan.

%\cite{sadat2012} menambahkan ruang publik memiliki peran fisik, ekologi, psikologi, sosial, politik, ekonomi, simbol, dan estetika. Seperti contoh, Jalan Malioboro di Jogyakarta memiliki peran bisnis dan rekreasi. Selain itu, dapat pula mempertunjukkan sejumlah elemen yang mengandung budaya dan peninggalan.

%munculnya ruang publik, its affect
Orang mulai memperhitungkan ruang publik ketika mereka mencari tempat yang dengan mudah dicapai. Ruang ini menjadi wadah dalam kegiatan-kegiatan sehari-hari masyarkat. Seperti contohnya jalan yang orang lintasi setiap hari. Alun-alun yang orang gunakan untuk berolahraga secara rutin. Munculnya aktivitas-aktivitas pada ruang publik memberikan kontribusi terhadap kualitas kehidupan perkotaan dan mencerminkan keadaan masyarakat sekitar.

%Ruang publik memiliki elemen-elemen fisik yang disukai orang.
%Kebanyakan penelitian menunjukkan ruang publik menarik lebih banyak orang karena memiliki fasilitas yang memadai yang mana berguna untuk relaksasi, olahraga, dan interaksi dengan kerabat. \cite{uliah} menyebutkan bahwa orang lebih menyukai ruang publik dengan \textit{setting} binaan daripada hanya \textit{setting} yang alami tanpa fasilitas yang cukup. Diluar daripada preferensi orang, muncul beragam ruang publik di sejumlah titik di sebuah kota khususnya di Parepare.
%Keberagaman fungsi dari ruang publik tersebut merupakan sebab dari keanekaragaman penduduk sekitar dan kebutuhan mereka terhadap ruang \citep{ahmadi2009}

\cite{ahmadi2009} menegaskan bahwa ruang publik memiliki karakter yang berbeda. Karakter tersebut menjelaskan tentang apa yang ada pada lingkungan fisik dan sosial \citep{dougherty2006}. Sementara \cite{hartanti2014} menyebut elemen fisik, aktivitas dan suasana sebagai sebuah pembentuk karakter. Meskipun demikian, sejumlah penelitian menemukan karakter adalah hasil dari permintaan dalam pemenuhan kebutuhan dan cerminan dari keadaan masyarakat sekitar.

%Karakter dapat menciptakan identitas suatu kota \citep{hartanti2014}.

%Menurut \cite{hartanti2014}, karakter dapat membentuk identitas suatu kota.

Parepare memiliki segudang ruang publik dengan karakteristik masing-masing. Seperti contoh, Taman/ Alun-alun Andi Makkasau, Pantai Parepare, Taman Syariah, Tepi Laut Tonrangeng, dan lain-lain.
Akhir-akhir ini, kebanyakan perhatian ada pada ruang publik di pesisir laut dimana menjadi fokus penelitian ini.
Festival Salo Karajae menandai masih berdetaknya jantung pesisir laut Parepare. Ribuan orang berbondong-bondong datang ke tepi laut tonrangeng untuk menyaksikan beragam pertunjukan dalam sepekan.
Ruang publik ini, menurut \cite{hartanti2014},   menjelaskan identitas suatu kota melalui karakter fisiknya.
Ini disebabkan oleh orang menikmati suatu kota dengan melihat-lihat, merasakan, dan memahami informasi berupa jalan setapak, lanmark, pepoohonan, laut, dan elemen lainnya.
Terkadang hubungan antara ruang publik dan kota adalah sangat erat. Orang menandai bahwa mereka pernah ke kota Parepare dengan berfoto ria di ruang publik BJ Habibie. Ruang publik, namanya, karakternya, telah menjadi identitas kota. Alhasil, elemen-elemen ruang sebaiknya didesain dengan baik agar menunjukkan karakter dan identitas yang sepatutnya diterima oleh pengguna \citep{hartanti2014}.

Dengan sejumlah ruang publik, Parepare mengusung gagasan kota Cinta BJ Habibie. BJ habibie adalah presiden ketiga Indonesia yang lahir di Parepare.
Kontribusi beliau terhadap Indonesia menjadi inspirasi kota Parepare. Gagasan ini seyogyanya merupakan identitas baru kota Parepare.
Identitas suatu tempat, menurut Kevin Lynch, merupakan dasar dalam mengenali sesuatu sebagai suatu hal yang terpisah \citep{hartanti2014}. Berdasarkan kamus \textit{Webster's Ninth New Collegiate}, Identitas adalah karakter atau kondisi pembeda dari seseorang atau benda.
Begitu juga, menurut \cite{hartanti2014}, identitas kota dapat dibentuk melalui karakter yang ada didalamnya. Itu dapat melalui bentuk fisik atau penampilan fisik. Akan tetapi, dapat pula melalui konsep kemampuan pencitraan \textit{(imageability)} yang terdiri atas identitas, struktur dan makna \citep{lynch1984}.

\cite{hartanti2014} menjelaskan identitas dapat menjadi syarat lingkungan yang bagus. Dia menambahkan membentuk identitas kota bakal memperkokoh perbedaannya, menciptakan kegiatan atau penanda khusus, dan mendukung sejumlah motif tertentu. Seperti contoh festival salo karajae yang telah menjadi sebuah acara yang istimewa dan unik di tepian laut. Acara tersebut telah berlangsung dalam beberapa tahun terakhir dan terbilang sukses. Berbagai macam pertunjukkan diadakan misalnya pemetaan video di laut, lomba perahu hias, dan lain-lain.

Penelitian mengenai karakter ruang publik telah banyak dilakukan, namun hanya sedikit yang melakukannya di area pesisir laut. Keanekaragaman karakteristik yang timbul akibat berbatasan dengan laut merupakan nilai tambah terhadap ruang publik. Apalagi daerah ini sangat berkontribusi terhadap perkembangan kota \citep{hussein2014}. \cite{hussein2014} menjelaskan kesuksessan kota membawa masyarakat ke pesisir laut bergantung pada ruang publik disana. Dengan begitu, penelitian ini akan menjelaskan karakter dari ruang publik dalam membentuk identitas kota Parepare.

\section{Rumusan Masalah}

%reform identtias again
Identitas kota yang sukses berasal dari pencitran yang kuat dan bebas \citep{oktay2002}. Kebanyakan itu dikarakterisasi oleh ruang publik. Sementara, ruang publik yang paling menonjol saat ini berada di kawasan pesisir laut. Kawasan ini adalah tempat yang menakjubkan dan fleksibel \citep{hussein2014}. Menurut \cite{oktay2002}, dalam pencarian identitas kota pada kontek kota yang berubah, menemukan ruang publik mestinya mencakup pengertian luas seperti kegiatan orang, pengalaman, dan hubungan dengan ruang publik itu sendiri.

Daerah-daerah pesisir laut memang memiliki potensi yang luar biasa. Parepare sebagai kota yang berbatasan dengan laut ini memanfaatkan kesempatan tersebut untuk membangun sejumlah ruang publik pesisir laut. Pembangunan tersebut terbilang drastis dimana dalam kurung waktu periode jabatan politik menghasilkan dan merombak tidak hanya satu ruang publik. Pembangunan tersebut pertama kali terjadi pada tahun 2011. Ini merenovasi ruang publik Pantai Senggol yang sekarang bernama Pantai Parepare. Itu kemudian disusul oleh pembangunan dari nol ruang publik lainnya. Disamping itu, pembangunan juga menyebar ke pusat kota.

Tidak seperti kota-kota lainnya, Parepare berangkat dengan identitas yang belum dikembangkan. Banyak perkotaan yang telah memiliki identitas yang kuat sejak lama. Seperti contoh kota bogor, kota dalam negeri, memiliki identitas kota yang berkaitan dengan tumbuh-tumbuhan \citep{hartanti2014}. Selain itu, ada kota Cyprus utara, kota luar negeri, memiliki identitas kota lama \citep{oktay2002}. \cite{oktay2002} menjelaskan identitas itu terdiri dari sejumlah karakteristik atau elemen-elemen ruang publik yang dapat teridentifikasi.

Adapun identitas kota Parepare yang diusung belum lama ini adalah kota cinta. Itu memicu pembangunan ruang publik bernama BJ Habibie dan Ainun. Satu yang paling menonjol adalah monumen BJ Habibie yang terletak di taman Andi Makkasau. Dua diantaranya berada di pesisir laut yaitu gedung serbaguna dan rumah sakit BJ Habibie yang dekat dengan tonrangeng \textit{riverside}.
Meskipun pembangunan ini mengisyaratkan \textit{``kota cinta"} BJ Habibie dengan sejumlah penamaan atau patung publik di ruang publik, itu tidak berarti ruang publik secara otomatis dapat membentuk identitas perkotaan yang diharapkan.

Karakter-karakter yang dimiliki tiap ruang publik memainkan peran penting dalam membentuk identitas perkotaan \citep{oktay2002}.
\cite{lynch1984} menjelaskan karakter ruang publik memperhatikan elemen-elemen yang tertata dan hubungannya satu sama lain serta membentuk suatu karakter pembeda,
%\cite{lynch1984} menjelaskan setiap elemen-elemen yang tertata dan berhubungan satu sama lain akan membentuk suatu pembeda atau perbedaan yang diterima \citep{rapoport1990},
yang mengindikasikan suatu identitas \citep{hartanti2014}. \cite{hartanti2014} menegaskan bahwa pembeda adalah proses terpenting dalam mengenal suatu identitas tempat. Ini berhubungan dengan  bentuk fisik, aktivitas dan suasana. Berdasarkan uraian sebelumnya, penelitian ini mempelajari karakter ruang publik di pesisir laut. Maka penelitian ini akan menjawab sejumlah pertanyaan penelitian sebagai berikut:

\begin{enumerate}
\item Apa karakter pembeda diantara ruang publik pesisir laut yang membentuk identitas kota Parepare?
\item Bagaimana hal tersebut mempengaruhi identitas untuk desain yang lebih baik?
\end{enumerate}

\begin{comment}
Karakter-karakter tersebut sebaiknya menimbulkan suatu pembeda

Karakter-karakter tiap ruang publik harus memiliki elemen yang sama yang menggambarkan karakter.

Satu diantara

%Itu merupakan bukti bahwa Parepare mencoba untuk memperkenalkan identitas kota cinta BJ habibie.
Mereka membangun patung atau monumen bj habibie, memberi nama setiap bangunan dengan bj habibie. akan tetapi karakteristik tidak hanya bermaksud pada penamaan atau bla bla, tetapi pengalaman, makna atau bla bla bla bahkan sense of place.
Meskipun ini bersifat simbolis,
kota seharusnya berisi karakteristik lokal sehingga kota memiliki suatu pembeda.

\end{comment}
\section{Tujuan Penelitian}
Penelitian ini bertujuan untuk mempelajari karakter-karakter dari ruang publik pesisir laut. Karakter tersebut dijelaskan dalam konteks elemen-elemen ruang, kegiatan orang, pengalaman, dan hubungan dengan ruang publik. Sehingga penelitian ini dapat menjelaskan identitas yang dimiliki kota Parepare. Menurut \cite{oktay2002}, identitas mampu membangun ketertarikan individu pada suatu tempat dan singgah dalam waktu lama. Kemampuan tersebut merupakan suatu tanda ruang publik pesisir yang berhasil \citep{hussein2014}.

Peneliti berharap hasil penelitian ini dapat digunakan dalam mengenal identitas kota Parepare.
Dengan begitu, pembangunan kota parepare dapaat mengacu pada identitas agar semakin kuat.

\begin{enumerate}
    \item Untuk mempelajari karakter pembeda diantara ruang publik pesisir laut dalam membentuk identitas kota.
    \item Untuk mengetahui pengaruh identitas dalam menciptakan desain yang lebih baik.
\end{enumerate}

\section{Metode penelitian}
Penelitian ini bertujuan untuk mempelajari karakter pembeda diantara ruang dalam membentuk identitas kota.




\begin{comment}

Akhir-akhir ini, Parepare berfokus pada pembangunan kepariwisataan.
Fokus ini membentuk gagasan kota cinta BJ habibie.


Gagasan tersebut untuk mendukung fokus Parepare sebagai kota Pariwisata. Sebagai wujud

%Parepare merupakan kota dengan sebutan kota cinta BJ habibie.
%Gagasan tersebut baru saja tercipta mengingat kota mulai berfokus pada aspek pariwisata.

Belakangan ini, Parepare berfokus pada pembangunan kepariwisataan. Fokus ini menciptakan identitas kota cinta BJ habibie. Alhasil, banyak bangunan publik yang bernama BJ. Habibie. Bangunan-bangunan tersebut adalah monumen, bangunan serba guna, rumah sakit, masjid terapung, bahkan stadium. Selain itu, terdapat pula ruang publik lainnya seperti pantai parepare, pantai cempae, dan taman syariah. Bangunan-bangunan tersebut menciptakan identitas kota Parepare melalui karakter-karakter yang dimiliki.

Pembangunan ruang publik akhir-akhir ini berfokus pada pesisir laut. Pesisir laut adalah area yang terbentang 30 meter dari garis pantai ke daratan. Adapun ruang publik tersebut adalah tonrangeng \textit{riverside}, masjid terapung, taman mattirotasi, pantai parepare, dan pantai cempae. Menurut \cite{uliah2022}, meningkatnya perhatian terhadap ruang publik merupakan hasil dari permintaan masyarakat terhadap ruang di perkotaan.

Pembangunan tempat wisata di pesisir ini terbilang masih cukup anyar.

kualitas ruang publik di pesisir laut ditinjau dari lima dimensi ruang publik.



Parepare merupakan kota dengan peringkat keempat memiliki garis pantai terpanjang \citep{uliah2022}. Sepanjang garis tersebut, terdapat sejumlah ruang publik. Ruang publik tersebut adalah Pantai Parepare, Taman Mattirotasi, Masjid Terapung, dan lain-lain.


Parepare memiliki sejumlah ruang publik, satu diantaranya adalah taman Andi Makassau yang telah melewati sejumlah revitalisasi. Revitalisasi yang terakhir terbangun di awal tahun 2022. Itu memberikan sejumlah fasilitas untuk olahraga, kesehatan, dan fasilitas yang meningkat seperti bangku, toilet, panggung, dan rumput yang berkualitas.
Kesuksessan taman ini dalam menarik pengunjung menandakan keadaan dari masyarakat yang aktif berolahraga dan kebugaran. Ruang publik lainnya adalah Pantai Parepare, alih-alih menyediakan jalur jogging, ruang ini memberikan tempat untuk berenang, makan, bahkan berinteraksi yang banyak. Kedua ruang tersebut tampaknya memberi warna terhadap karakter mereka masing-masing.


Tidak hanya taman andi makasssau, penekanan pada peran ekonomi, estetika, dan simbolis merupakan karakter ruang publik secara umum.



% Perubahan  pada karakter ruang publik  dipengaruhi oleh perubahan gaya hidup dan aktivitas pola dari pengguna. , the morphology and characteristic



%Elemen mungkin dapat menggambarkan karakteristik suatu ruang publik.
%\cite{montgomery1997} mengindikasikan elemen dapat mewakili suatu nilai tatanan yang lebih tinggi. Seperti contohnya tempat berkumpul yaitu patung BJ. Habibie menyimbolkan nilai budaya, sejarah, sosial dan politik masyarakat \citep{akkar2007}. Sehingga ruang publik memiliki karakteristik yaitu sebagai peran simbolis terhadap suatu kota.


% muncul beragam desain elemen yang mengubah peran ruang publik
k

%
%Penelitian tentang elemen-elemen ruang publik masih kurang mengaitkan peran raung publik terhadap perkotaan.

Penelitian tentang peran elemen-elemen ruang publik

terhadap suatu kota

%\cite{akkar2007} menjelaskan bahwa ruang publik menggambarkan sebuah identitas pada suatu kota.
%Diantara semua peran yang dapat melekat pada ruang publik, penting untuk memahami citra apa yang disampaikan ruang publik terhadap suatu kota. Citra ini dapat tersampaikan melalui elemen-elemen ruang yang terdapat di ruang publik. Sebagai contoh, tempat berkumpul yaitu patung BJ. Habibie menggambarkan nilai budaya, sejarah, sosial dan politik mereka \citep{akkar2007}. Sehingga bisa dikatakan ruang publik merupakan simbol terhadap kelompok dari masyarakat.


Hal itu karena ruang publik mungkin mewakili elemen yang mengandung nilai dari tatanan yang dianut \citep{akkar2007}. Seperti contohnya pertemuan suci atau simbolis. Patung habibie sebagai contoh dari perwakilan dari pertemuan yang simbolis. Patung ini ditempatkan disudut taman Andi Makassau. Patung ini sangat populer diantara pelancong dan pengunjung dari Parepare dan terbangun pada beberapa tahun lalu. Nama dari patung tersebut terinspirasi oleh nama presiden republik Indonesia. Patung tersebut secara khusus dibangun untuk menarik pengunjung dari seluruh provinsi sulawesi selatan dan diluar serta memperkuat citra Parepare sebagai kota pariwisata.







\end{comment}
%--------------------------------------------------------------------------------------
% Draft
%
%--------------------------------------------------------------------------------------
\begin{comment}

\section{Research Objective and Questions}
Typical New Public Space in Parepare City

\subsection{Research Objectives}

\subsection{Research Questions}



\section{Introduction of Research Proposal}

\subsection{Background of research}

%the emerge of public space
Public space has become a provocative discussion on urban planning in the last 100 years. It was led by an increasing of population in the city which hope for the open space.
The increasing population is marked by the massive development of residence, the density of highway, the rising of business and economy sphere that has occurred in Parepare. It put stress on individual to go through their lives in every single day. People start seeking a place where they can heal from the hustle and bustle of daily routine and reduce the stress as a result of urban living.

%public space in the city
Public space is a place that provide any kind of individual or group  activities in pursuit of positive things \citep{sadat2012}.\cite{sadat2012} explain a large square, waterfront, street, and traditional bazaar as a form of a public space.
People might find public space as a place for recreation, working, business, social, cultural and heritage. \cite{sadat2012,akkar2007} have collected the function of public space broadly consist of physical, ecological, physiological, social, political, economic, symbolic and aesthetics roles.
Moreover, one public space can extend to other purpose. For example, Maliboro corridor in Jogjakarta have business and recreation function. Beside that, it also depict some cultural and heritage element that until now are preserved.

People begin to notice the importance of public space in the city when they cannot find the relaxing, calm, and comfortable place nearby and easily. Sometimes, the importance become more striking when the place was used in the daily basis. It can be a street, or corridor, bazaar, plaza surrounded by building where people especially workers, pedestrian, and picnicker are doing their own thing, such as shopping exactly at front store, talking while walking, or just passing through for work.
The activities took place around strengthen the importance of public space. But they also show contribution determining the quality of urban lives and mirror the state of neighborhood.

% people views
Public space then have their own characteristic that people are most enjoyed. Researcher found that public space attract more people because it has adequate facilities which they can use it for relaxation, sport and interaction with their friends.
\cite{uliah} argue that people prefer public space that have outstanding setting rather than just natural setting or without many facilities. Regardless the preference of most people, multiple variation of public space are emerging in few place of the city.
This may because of the diversity dwellers around those public space and their unique needs towards a space \citep{ahmadi2009}.

\cite{ahmadi2009} say that the public space has a different value. Although it has unique value, sometimes adjacent public spaces express the same common purpose. One of significant purpose of most public space is economic \citep{akkar2007}, which has evolved since long time. But, some of them is following the present-day demands like the physiological and physical which are results of hectic urban lives. So those different purpose, one of them will adhere in the identity of the city.

Parepare has owned few of public space, one of the greatest is Andi Makassau Park which has undergone some revitalization. Its recent revitalization happen in an early 2022. It brings some facilities for sport, health and other improved facilities such as bench, toilet, stage, maintained grass.
This improvement has altered the city identity. Furthermore, specifically, it increase the physiological well-being through the availability of facility. It shows that more or less this public space has pictured some states of society who are active in sports and fitness. In addition to that, This park also successfully have arranged several events include Independence day ceremony, gymnastic, and other events which were attracted vendors to do economic activity. Therefore, the place has shown other than just physical or physiological role of urban space, yet it also shows an economy part of Andi Makassau Park.

The role Andi makassau play as public space might spread across other public space. For instance, the near public psace like Parepare Beach, Tonrangeng Waterfront, and Syariah Small Park. These public space have a number of roles similar to  Andi Makassau Park. The role they have played in the city will have contribution to the urban people. For example, instead, Parepare beach provide a jogging track, it gives a place for visitor to swimming, eating and even just having interaction. Then places eventually bring some benefit to people.

Among all the roles that can be played public space, it is important to understand a general image that public space give to the city. One of the reason is that public space might represent the element which is containing a value of higher order \citep{akkar2007}.  For example, sacred or symbolic meeting. Habibie sculpture comes up as an example of symbolic meeting. This sculpture is placed in the corner of Andi Makassau Park. It has been popular among the tourism and visitor of Parepare city and built around a decade.
The name of statue was inspired by the name of the third Indonesia president. The sculpture is purposely built to attract visitor from all around South Sulawesi province and outside and also strengthen the image of Parepare as tourism city.




%The role of developing waterfront in improving identify of parepare town as recreational town

\begin{enumerate}
\item The proposed research seeks to analyze the importance of developing three waterfronts landscape in arrangement of the place identity. Therefore, this research project will investigate what are characteristic of waterfront landscape in Parepare town identified in the study area.
\item This research proposal will investigate the character of developing waterfront landscape of recreational Parepare city from place, contemporary, and recreational significance point of views.
\item The proposed study will focus on creating recomendation to improvement of waterfronts development in enhancing the identity of Recrational City.

\end{enumerate}

\subsection{Research Question}

\subsubsection{Draft}
What are the characterstics of Parepare waterfronts related to recreational city context?

Waterfront is one of the interesting place in the city these days. Although it had experienced several function changes, like industrial, cultural, and business function, many cities centralize its development to public open space in the waterfront, including Parepare. Waterfront in Parepare has improved significantly in the past decade. Its development was one of the result of city major's effort in order to make Parepare become more recreational. This improvement is also caused by the increasing of population density. Many new settlements have been growing in the suburb area. This phenomenon has raised the need of better public open space in the city center.

Cities become interest with its functional to bring public space to attract people from city centre.

Until it become the public space,

In the past decade, Parepare is focus on infrastructure development.

Tepi laut menjadi sumber kehidupan pada sebuah daerah yang memiliki garis pantai yang panjang.

Selayaknya air, tepi laut menjadi sumber kehidupan pada sebuah daerah yang memiliki garis pantai yang signifikan. Tidak seperti sebuah daerah yang tidak memiliki garis pantai, terpaksa menggantungkan kehidupan pada sumber lainnya seperti.

Tepi laut adalah sebuah tempat yang berada di pinggiran laut.

menyebut terbentang 30 meter dari garis pantai.



\section{Tipikal Ruang Publik yang Terbaru di Kota Parepare}
%munculnya ruang bpublik
Ruang publik telah menjadi momok pembicaraan dalam perencanaan kota dalam 100 tahun terakhir ini. Satu alasannya adalah peningkatan populasi yang kian membesar pada sebuah kota yang mengharapkan ruang yang terbuka. Peningkatan populasi tersebut ditandai dengan adanya pembangunan perumahan yang besar-besaran, keramaian di jalan raya, meningkatknya sektor ekonomi dan perekonomian  seperti yang terjadi di Parepare. Peningkatan populasi ini memberi tekanan pada setiap individu dalam menjalani kehidupan sehari-hari.
Orang-orang kemudian berbondong-bondong mencari tempat untuk penyembuhan \textit{(healing)} untuk keluar dari hiruk pikuk rutinitas sehari-hari dan menghilangkan tekanan akibat dari suasana perkotaan.

%ruang publik pd perkotaan
Ruang publik adalah sebuah tempat yang mewadahi segala aktivitas perorangan atau perkelompok dalam mencari kesenangan dan hal positif lainnya \citep{sadat2012}.
Diluar dari ruang publik, banyak orang memilih untuk sepenuhnya berekreasi ditempat lain yang menawarkan rekreasi khusus seperti pantai, waterbom, \textit{themepark}, pegunungan, dan kampung. Namun tidak sedikit yang memilih \textit{healing} secara sederhana dalam sebuah ruang publik perkotaan yang menawarkan kehidupan perkotaan yang dibungkus oleh aktivitas yang menyenangkan.
Ruang terbuka publik (rtp) tercipta untuk mereka yang mencari pusat rekreasi  dengan akses yang gratis, akses yang mudah dan tidak memerlukan banyak waktu tempuh. Sehingga ruang publik menjadi primadona bagi mereka yang tidak menyukai kerumitan untuk berwisata. Meskipun ruang publik terlihat sederhana, banyak potensi-pontesi ruang publik yang bernilai tinggi apabila orang menjelajahinya lebih dalam.

%potensi ruang publik
Ruang publik sepatutnya tidak terikat oleh batas ruang dan waktu. Itu memberi ruang untuk setiap golongan dalam jumlah berapapun untuk memaksimalkan ruang publik sebagai tempat untuk berekreasi. Walaupun sifatnya satu ruang untuk semua orang, dimana kita berbagi dengan orang asing secara sukarela \citep{sadat2012}. Ruang publik berkesempatan untuk mengakomodasi segala bentuk aktivitas didalamnya. Sehingga tujuan dominan dari ruang publik yaitu mensejahterakan masyarakat dapat teruwujud.

Banyak yang bergumam tentang ruang publik yang seutuhnya.

%rtp secara umum

% Peningkatan di parepare
Kota ini dalam beberapa tahun terakhir menunjukkan peningkatan dalam segi pelayanan bidang dan jasa. Sehingga mendorong terjadinya urbanisasi yang cukup tinggi.

Kualitas hidup seseorang ditentukan oleh elemen ruang publik dari sebuah kota.

Meningkatnya pembangunan perumahan di kota Parepare, mendorong ketersediaan ruang publik yang memenuhi kebutuhan masyarakat Parepare.


%--------------------
%Therefore, project of research will investigate what are characters and elements of developing waterfronts in the context of Recreational Parepare town.
%\item This research proposal examine the impact of developing waterfronts landscape towards the growing recreational city of Parepare. Then, classify what are growing waterfronts landscape charateristic that can be identified in the area of study.
%\item The proposed research will focus on the element of growing waterfronts landscape that can be maintainted to increase the value of recreational Parepare town.

\subsection{Research Questions}

The proposed research seeks to analyze developing three waterfronts landscape in arrangement of the place identity. Therefore, project of research will investigate what are characters and elements of developing waterfronts in the context of Recreational Parepare town.
This research proposal examines the impact of developing waterfronts landscape towards the growing recreational city of Parepare. Then, classify what are growing waterfronts landscape charateristic that can be identified in the area of study.
The proposed research will focus on the element of growing waterfronts landscape that can be maintainted to increase the value of recreational Parepare town.

% draft words
This research proposal is seeking to understand
The proposed research seeks to


Waterfront in Parepare has undergone rapid development.

How close bj habibie image on waterfront development parepare.

Transformation of waterfront parepare toward the city development.

Are the waterfront valued the same by people.

Waterfront character as identity representative of the Parepare city
* does waterfront should be designed following city branding.

what the connection of waterfront in parepare in terms of image of the city.

The role of devoloping waterfront in improving identify of parepare city as recreational town



this file powered by \citep{einstein} and \citep{latexcompanion}. It can also be found in \cite{knuthwebsite}.

\begin{figure}[htpb]
	\centering
	\includegraphics[width=0.8\textwidth]{placeholder}
	\caption{placeholder.jpg}
	\caption*{Sumber: Dokumen Pribadi, 2020}
	\label{fig:placeholder-jpg}
\end{figure}
\lipsum[1-3]

\subsection{Latar Belakang}

\lipsum[4-5]


\subsubsection{Definisi}

\lipsum[6-7]
\begin{table}[h] % [h] forces the table to be output where it is defined in the code (it suppresses floating)
	\caption{Example table.}
	\centering
	\begin{tabular}{l l r}
		\toprule
		\multicolumn{2}{c}{Name} \\
		\cmidrule(r){1-2}
		First Name & Last Name & Grade \\
		\midrule
		John & Doe & $7.5$ \\
		Richard & Miles & $5$ \\
		\bottomrule
	\end{tabular}
\end{table}



\subfile{subfiles/subfile.tex}

\section{Metodologi Penelitian}

\lipsum[15-19]

%\begin{figure}[htbp]
%\centering
%\begin{subfigure}{6cm}
%\centering\includegraphics[width=5cm]{placeholder.jpg}
%\caption{}
%\end{subfigure}%
%\begin{subfigure}{6cm}
%\centering\includegraphics[width=5cm]{placeholder.jpg}
%\caption{}
%\end{subfigure}\vspace{10pt}
%
%\caption{Lorem Ipsum}
%\label{}
%\end{figure}

\end{comment}
%----------------------------------------------------------------------------------------
%	BIBLIOGRAPHY
%----------------------------------------------------------------------------------------

\bibliographystyle{apalike}

\bibliography{biblio.bib}


\end{document}
